
\section{Materials and Methods}

\subsection{Small Radio Telescope (SRT)}
All measurements were performed using the small radio telescope located on top of the ExWi building.
It has a parabolic dish with a diameter of \SI{4}{m} and a feedhorn antenna. This causes main and side lobes, where the shape of the main lobe can be approximates as Gaussian.

\subsubsection{Signal Processing}
TODO: insert sketch with the process

TODO: explain signal process chain

Therefore output of the Fourier transform are therefore amplitude densities with units of [\si{\volt/\hertz}]. This value is proportional to the power density and its integration is proportional to the power.
This can lead to confusion if we denote axes in figures with units of Volts even though their physical interpretation should be as a value of Power.
For the rest of this report we will denote the unit of this amplitude in arbitrary units [\si{a.u.}] and the spectral amplitude density with units of [\si{a.u./\hertz}].
We will also use the symbol $P$ to denote such values to better convey the physical meaning.

\subsection{Lab View Tool}
sketch, description


\subsection{Measurement data and Code}
All data and code used in this report are available in a public repository\footnote{\url{https://github.com/SecretGmG/lab/tree/main/srt}}.

Error propagation and data fitting were performed using the public-domain module PhysicsTool\footnote{\url{https://github.com/SecretGmG/PhysicsTool}}.