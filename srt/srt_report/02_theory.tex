\section{Theory}
\subsection{Exercises}
\subsubsection[Origin of the HI-Line]{Origin of the \ce{H_I}-Line}

\subsubsection{Angular resolution of the SRT}\label{sec:ang_res}
Using the equation relating the effective area $A_e$ and the wavelength $\lambda$ \cite[p. 149]{wilson}
\begin{equation}
    A_e \Omega_A = \lambda^2
\end{equation}
and the definitions of the effective area and the geometric area \cite[p. 148]{wilson}
\begin{equation}
    A_e = \eta A_g \qquad A_g = \pi \left( \frac{d}{2} \right) \label{eq:A_e}
\end{equation}
we can then solve for the effective antenna area.
\begin{equation}
    \Omega_A = \frac{\lambda^2}{A_g \eta} \label{eq:Omega_A}
\end{equation}


To compute an approximation of the full width at half maximum $\theta$ (FWHM) we can approximate the aperture as gaussian \cite[p. 2]{srt} and note the following properties.
\begin{equation}
    \Omega_A \approx \int_{4\pi } d\Omega\, \exp{\left( -\frac{\theta^2}{2\sigma^2} \right)} \approx 2\pi \int_0^{\infty} d\theta \, \theta \exp{\left( -\frac{\theta^2}{2\sigma^2} \right)} = 2\pi \sigma^2
    \label{eq:gauss_integral}
\end{equation}
\begin{equation}
    \exp{\left( -\frac{(\theta/2)^2}{2\sigma^2}\right)} = \frac{1}{2} \quad \implies \quad \theta = 2 \sqrt{2\ln{2}} \sigma \label{eq:FWHM}
\end{equation}
using equations \eqref{eq:gauss_integral} and \eqref{eq:FWHM} we can then solve for $\Omega_A$ in terms of $\theta$
\begin{equation}
    \Omega = \frac{\pi}{4\ln{2}} \theta \approx 1.133 \theta^2 \label{eq:Omega_A_fwhhm}
\end{equation}
Notice that this is consistent with \cite[p.178 (8.13)]{wilson}

Equating \eqref{eq:Omega_A} and \eqref{eq:Omega_A_fwhhm} and then solving for $\theta$ we get an expression for the expected FWHM.
\begin{equation}
    \theta = \frac{4}{\pi} \sqrt{\frac{\ln{2}}{\eta}}\frac{\lambda}{d}
\end{equation}

To compute the angular resolution we need the diameter of the SRT which is $d = \SI{4}{m}$ \cite[p. 4]{srt}, and the antenna efficiency which is $\eta = 0.5$ \cite[p. 2]{srt}.
Unfortunately we cannot reasonably estimate an error for the antenna efficiency $\eta$, we suspect however that the error of the antenna efficiency dominates here.
Therefore we will waive a proper error discussion at this point.

We can then obtain the values shown in table \ref{tab:ang_res}.
\begin{table}[H]
    \centering
    \begin{tabular}{rrrr}
        \toprule
        $\theta$ (FWHM) [\si{\degree}] & Wavelength $\lambda$ [\si{\degree}] & Frequency [\si{\giga \hertz}]\\
        \midrule
        \num{4.533} & \num{21.11} & \num{1.420}\\
        \num{0.688} & \num{0.30} & \num{10.000}\\
        \bottomrule
    \end{tabular}
    \caption{Angular resolution of the SRT at different frequencies}
    \label{tab:ang_res}
\end{table}
\subsubsection{Antenna, brightness and noise temperature}

\subsubsection{Temperature conversion}

\subsubsection{Expected temperature of the sun}\label{sec:temp}
If we assume that the sun is a black body near the wavelength of \SI{1.42}{\giga\hertz} we expect it to have a brightness temperature of $T_{\odot} = \SI{5778}{\kelvin}$ \cite[p. 211]{ftb}.
In \cite[p.8-45 fig. 8-34]{kraus} we can see however TODO:

\subsubsection{Influence of elevation angle}


\subsubsection{Doppler shift}